\documentclass[12pt, a4paper]{article}

\usepackage[utf8]{inputenc}
\usepackage[spanish, es-tabla]{babel}

\usepackage[a4paper, margin=2.5cm]{geometry}
\linespread{1.2}

\usepackage{xcolor}
\usepackage{graphicx}
\usepackage{hyperref}
\usepackage{listings}
\usepackage{booktabs}

\hypersetup{
    colorlinks=true,
    linkcolor=blue!70!black,
    urlcolor=cyan!70!black,
}

\renewcommand{\lstlistingname}{Código}

\lstdefinestyle{mystyle}{
    backgroundcolor=\color{black!5},
    basicstyle=\ttfamily\footnotesize,
    breaklines=true,
    frame=single,
    framerule=0.5pt,
    rulecolor=\color{black!20},
    numbers=left,
    numberstyle=\tiny\color{gray},
    numbersep=5pt,
    captionpos=b,
}
\lstset{style=mystyle}

\begin{document}

\begin{titlepage}
    \centering
    \includegraphics[width=0.5\textwidth]{images/odoo_logo.png}
    
    \vspace{3cm}
    
    \Huge\bfseries
    Documentación Técnica: \\
    Modelos y vistas de los módulos de Odoo
    
    \vfill
    
    \Large
    \textbf{Autor:} Pedro José Meixús Belsol \\
    \vspace{0.5cm}
    \today
\end{titlepage}

\tableofcontents
\newpage

\section{Modificar lista de tareas}
En esta primera actividad vamos a modificar la lista de tareas básica, en mi caso utilizaré la que habia modificado levemente para la anterior práctica (incluye una fecha de inicio y otra de fin).

Lo primero que haremos será cambiar nuestra lista de tareas por una vista Kanban, para ello simplemente vamos a cambiar todo lo relacionado con "list" por "kanban" y añadiremos un bloque de código
"template" para indicar como se debe mostrar cada tarjeta del kanban. Quedaría de la siguiente manera:

\vspace{0.5cm}

\begin{lstlisting}[caption=Template para vista Kanban]
<templates>
    <t t-name="card">
        <div class="oe_kanban_card">
            <strong><field name="tarea"/></strong>
            <div>Prioridad: <field name="prioridad"/></div>
            <div>Urgente: <field name="urgente"/></div>
            <div>Realizada: <field name="realizada"/></div>
            <div>Creada: <field name="fecha_creacion"/></div>
            <div>Terminada: <field name="fecha_terminada"/></div>
        </div>
    </t>
</templates>
\end{lstlisting}

\vspace{0.5cm}

La vista deberia quedar algo similar a esto.

\vspace{0.5cm}

\begin{center}
    \includegraphics[width=0.8\textwidth]{images/vista_kanban.png}
\end{center}

\vspace{0.5cm}

Ahora nos falta añadir una vista de tipo calendario, donde usaré las fechas que ya cree en práctica anterior. Tendremos que añadir un nuevo "record" y dentro de este meteremos
un "calendar" donde le indicaremos las fechas de inicio y fin (obtenidas de los campos correspondientes), el color si es prioritaria y el nombre.

\vspace{0.5cm}

\begin{lstlisting}[caption=Template para vista Calendar]
<record model="ir.ui.view" id="lista_tareas.calendar">
    <field name="name">lista_tareas calendar</field>
    <field name="model">lista_tareas.lista_tareas</field>
    <field name="arch" type="xml">
        <calendar
            color="prioridad"
            date_start="fecha_creacion"
            date_stop="fecha_terminada"
            string="Calendario de Tareas">

            <field name="tarea"/>
            <field name="prioridad"/>
            <field name="urgente"/>
            <field name="realizada"/>

        </calendar>
    </field>
</record>
\end{lstlisting}

\vspace{0.5cm}

Cabe recalcar que en el 'act.window' debemos añadir tanto 'kanban' como 'calendar' para que muestre las vistas correctamente y sin darnos errores, de la siguiente manera:

\vspace{0.5cm}

\begin{lstlisting}[caption=Act window con tipos de vistas]
<record model="ir.actions.act_window" id="lista_tareas.action_window">
    <field name="name">Listado de tareas pendientes</field>
    <field name="res_model">lista_tareas.lista_tareas</field>
    <field name="view_mode">kanban,form,calendar</field>
</record>
\end{lstlisting}

\vspace{0.5cm}

Con todo esto hecho ahora deberiamos tener la opción de cambiar a vista calendario en la parte superior derecha y se vería tal que así:

\vspace{0.5cm}

\begin{center}
    \includegraphics[width=0.8\textwidth]{images/vista_calendar.png}
\end{center}

\vspace{0.5cm}

\newpage
\section{Ampliación módulo biblioteca de cómics}

En la segunda actividad vamos a modificar el módulo de biblioteca de cómics para que nos permita:

\begin{itemize}
    \item Gestionar socios (con nombre,apellido e identificador)
    \item Gestionar ejemplares de préstamo (con el socio al que se ha prestado, fehca de inicio de prestamo y fecha de devolución esperada)
    \item Restringir fecha de préstamo (no puede ser posterior al día actual) y fecha de devolución (no puede ser anterior al día actual)
\end{itemize}
 
Empezaremos por la gestión de los socios. Para esto crearemos el modelo 'socios.py' y el view correspondiente. En el modelo pondremos los campos mencionados anteriormente para lo socios, 
indicandole el nombre, si es requerido y una ayuda (por si el nombre no es suficientemente descriptivo). También tuve que añadir un 'sql.constraints' para hacer único el identificador.

\vspace{0.5cm}

\begin{lstlisting}[caption=Modelo de los socios]
    nombre = fields.Char(
        string='Nombre',
        required=True,
        help='Nombre del socio'
    )

    apellido = fields.Char(
    string='Apellido',
    required=True,
    help='Apellido del socio'
    )

    identificador = fields.Text(
        string='Identificador',
        required=True,
    )

    _sql_constraints = [
        ('identificador_unique', 'unique(identificador)', 'El identificador debe ser unico')
    ]
\end{lstlisting}

\vspace{0.5cm}

Para mostrar esto, añadiremos el view del socio como 'socio.xml', en el añadiremos toda la parte visual referente a los socios, como el boton para acceder desde el menú,
el formulario para crear socios o la lista donde se mostrará la información de estos.

Configuraremos la vista a nuestro gusto, siempre y cuando cumpla con lo pedido. En mi caso la vista principal quedó tal que así:

\vspace{0.5cm}

\begin{center}
    \includegraphics[width=0.8\textwidth]{images/vista_socios.png}
\end{center}

\vspace{0.5cm}

Y el formulario se ve así:

\vspace{0.5cm}

\begin{center}
    \includegraphics[width=0.8\textwidth]{images/formulario_socios.png}
\end{center}

\vspace{0.5cm}

Ahora para los ejemplares crearemos un nuevo modelo llamado 'ejemplar.py'. Declararemos los campos que vamos a necesitar para el ejemplar que serán: el nombre del ejemplar,
el id del comic, el id del socio, la fecha de prestamo, la fecha de devolución y el estado (disponible o no).

\vspace{0.5cm}

\begin{lstlisting}[caption=Modelo de los ejemplares]
        nombre = fields.Char(
        string='Nombre',
        related='comic_id.nombre',
        readonly=True
    )
    comic_id = fields.Many2one(
        'biblioteca.comic',
        string='Comic',
        required=True
    )
    socio_id = fields.Many2one(
        'biblioteca.comic.socio',
        string='Prestado a'
    )
    fecha_prestamo = fields.Date(
        string='Fecha de prestamo'
    )
    fecha_devolucion = fields.Date(
        string='Fecha prevista de devolucion'
    )
    estado = fields.Selection(
        [('disponible', 'Disponible'), ('prestado', 'Prestado')],
        string='Estado',
        compute='_compute_estado',
        store=True
    )
\end{lstlisting}

\vspace{0.5cm}

Después de esto vamos a controlar las fechas como se nos pide con 'api.constraints', utilizando un 'if' para comprobar si la fecha es anterior o posterior al día actual:

\vspace{0.5cm}

\begin{lstlisting}[caption=Comprobar fecha válida]
    @api.constrains('fecha_prestamo')
    def _check_fecha_prestamo(self):
        for record in self:
            if record.fecha_prestamo and record.fecha_prestamo > date.today():
                raise ValidationError('La fecha de prestamo no puede ser posterior al dia actual')
    
    @api.constrains('fecha_devolucion')
    def _check_fecha_devolucion(self):
        for record in self:
            if record.fecha_devolucion and record.fecha_devolucion < date.today():
                raise ValidationError('La fecha prevista de devolucion no puede ser anterior al dia actual')
\end{lstlisting}

\vspace{0.5cm}

Por último vamos a indicar el estado del ejemplar. Utilizaremos un campo computado teniendo como referencia el id del socio y la fecha de préstamo, en caso de tener un 
socio asignado (siendo al que se le ha prestado) y la fecha de préstamo indicada, se marcará como tal.

\vspace{0.5cm}

\begin{lstlisting}[caption=Indicar estado del ejemplar]
    @api.depends('socio_id', 'fecha_prestamo')
    def _compute_estado(self):
        for record in self:
            if record.socio_id and record.fecha_prestamo:
                record.estado = 'prestado'
            else:
                record.estado = 'disponible'
\end{lstlisting}

\vspace{0.5cm}

Antes de continuar, vamos a cambiar el .csv en la carpeta 'security' para después tener acceso y poder ver las funcionalidades creadas. En este archivo indicaremos el id, el nombre del permiso, el modelo al que se aplica, grupo de usuarios al que se aplica y los permisos que le daremos (lectura,escritura,creación,eliminación):

\vspace{0.5cm}

\begin{lstlisting}[caption=CSV de seguridad]
id,name,model_id:id,group_id:id,perm_read,perm_write,perm_create,perm_unlink
access_biblioteca_comic_user,biblioteca_comic_user,model_biblioteca_comic,base.group_user,1,1,1,1
access_biblioteca_categoria_user,biblioteca_comic_categoria_user,model_biblioteca_comic_categoria,base.group_user,1,1,1,1
access_biblioteca_socio_user,biblioteca_comic_socio_user,model_biblioteca_comic_socio,base.group_user,1,1,1,1
access_biblioteca_ejemplar_user,biblioteca_comic_ejemplar_user,model_biblioteca_comic_ejemplar,base.group_user,1,1,1,1
\end{lstlisting}

\vspace{0.5cm}

Con las funcionalidades lista podemos pasar a hacer la vista que llamaremos 'biblioteca\_comic\_ejemplar.xml'. Indicaremos lo de siempre, como se debe mostrar en el menú, como 
será el formulario para crear nuevos ejemplares y como se verá la lista de ellos. En mi caso la vista principal se ve de la siguiente manera:

\vspace{0.5cm}

\begin{center}
    \includegraphics[width=0.8\textwidth]{images/vista_ejemplar.png}
\end{center}

\vspace{0.5cm}

Y el formulario de creación de ejemplares es tal que así:

\vspace{0.5cm}

\begin{center}
    \includegraphics[width=0.8\textwidth]{images/formulario_ejemplar.png}
\end{center}

Ahora que tenemos todo hecho vamos a probar a crear un ejemplo para cada modelo y ver los diferentes campos que tenemos para rellenar.

Por un lado en los cómics vamos a crear un ejemplo de Superman:

\vspace{0.5cm}

\begin{center}
    \includegraphics[width=0.8\textwidth]{images/creacion_comic.png}
\end{center}

\vspace{0.5cm}

Esta sería la creación de una categoría:

\vspace{0.5cm}

\begin{center}
    \includegraphics[width=0.8\textwidth]{images/creacion_categoria.png}
\end{center}

\vspace{0.5cm}

Añadiremos un nuevo socio:

\vspace{0.5cm}

\begin{center}
    \includegraphics[width=0.8\textwidth]{images/creacion_socio.png}
\end{center}

\vspace{0.5cm}

Y por último crearemos un ejemplar del cómic que hemos creado y se lo llevará prestado el usuario que hemos creado:

\vspace{0.5cm}

\begin{center}
    \includegraphics[width=0.8\textwidth]{images/creacion_ejemplar.png}
\end{center}

\vspace{0.5cm}


\newpage
\section{Creación módulo de un hospital}

En esta actividad vamos a crear un módulo para un hospital en el que se tenga en cuenta lo siguiente:

\begin{itemize}
    \item Pacientes: Tienen nombre, apellidos y síntomas
    \item Médicos: Tienen nombre, apellidos y número de colegiado
    \item Consulta: Un médico atiende a un paciente.
    \item Relaciones: Un paciente podrá ser atendido por varios médicos y un médico podrá atender a varios pacientes
\end{itemize}

Con esto en cuenta, empezaremos creando la base del módulo con el comando 'scaffold', primero accedemos a nuestro contenedor con odoo, ejecutamos el comando poniendo el nombre  que queremos para nuestro módulo (en este caso hospital) y le indicamos la ruta donde debe crearlo, a mayores le pondremos todos los permisos con 'chmod' por si acaso:

\vspace{0.5cm}

\begin{lstlisting}[caption=Scaffold para módulo]
    docker exec -it odoo18 bash
    odoo scaffold hospital /mnt/extra-addons
    chmod 777 -R /mnt/extra-addons/hospital
\end{lstlisting}

\vspace{0.5cm}

Es importante añadir en \_\_init\_\_.py el import de los modelos, podemos hacerlo ahora o después de crearlos, es indiferente.

\vspace{0.5cm}

\begin{lstlisting}[caption=Imports en init.py]
from . import models
from . import paciente
from . import medico
from . import consulta
\end{lstlisting}

\vspace{0.5cm}

Con la base preparada, vamos a empezar con la creación de los modelos. Usaremos un modelo para médicos, otro para pacientes y un tercero para consultas. Empezemos por el de médicos:

En este módulo vamos a indicar los campos que tendrá, que serán el nombre (nombre y apellidos), número de colegiado y la relación de la consulta (esta relación será un 'One2Many' ya que un médico puede tener X consultas, pero cada consulta la realiza un único médico)

\vspace{0.5cm}

\begin{lstlisting}[caption=Modelo médico]
    from odoo import models, fields

    class Medico(models.Model):
        _name = 'hospital.medico'
        _description = 'Medico'

        name = fields.Char(string="Nombre y apellidos", required=True)
        numero_colegiado = fields.Char(string="Numero de colegiado", required=True)

        consulta_ids = fields.One2many(
            'hospital.consulta',
            'medico_id',
            string="Consultas"
        )
\end{lstlisting}

\vspace{0.5cm}

En el modelo del paciente tendremos algo muy similar, simplemente cambiaremos el campo de número de colegiado por el de síntomas que se nos pide (La relación sigue siendo igual).

\vspace{0.5cm}

\begin{lstlisting}[caption=Modelo paciente]
    from odoo import models, fields

    class Paciente(models.Model):
        _name = 'hospital.paciente'
        _description = 'Paciente'

        name = fields.Char(string="Nombre y apellidos",required=True)
        sintomas = fields.Text(string="Sintomas")

        consulta_ids = fields.One2many(
            'hospital.consulta',
            'paciente_id',
            string="Consultas"
        )
\end{lstlisting}

\vspace{0.5cm}

Por último, el modelo de la consulta tendrá 'la otra parte' de las relaciones para médicos y pacientes (siendo 'Many2one'), además de el diagnóstico y una fecha.

\vspace{0.5cm}

\begin{lstlisting}[caption=Modelo consulta]
from odoo import models, fields

class Consulta(models.Model):
    _name = 'hospital.consulta'
    _description = 'Consulta Medica'

    paciente_id = fields.Many2one(
        'hospital.paciente',
        string="Paciente",
        required=True
    )

    medico_id = fields.Many2one(
        'hospital.medico',
        string="Medico",
        required=True
    )

    diagnostico = fields.Text(string="Diagnostico")
    fecha = fields.Datetime(string="Fecha", default=fields.Datetime.now)
\end{lstlisting}

\vspace{0.5cm}

Después de terminar los modelos vamos a diseñar las vistas. En mi caso utilizaré 'list' (para mostrar los existentes) y 'form' (para la creación de nuevos). Empezemos por los médicos.

Creamos 'medico.xml', en el indicaremos lo que se mostrará en la lista y los campos para el formulario. Mi vista es la siguiente:

\vspace{0.5cm}

\begin{center}
    \includegraphics[width=0.8\textwidth]{images/vista_medico.png}
\end{center}

\vspace{0.5cm}

Después, tenemos la vista para los pacientes, que será prácticamente igual a la de los médicos:

\vspace{0.5cm}

\begin{center}
    \includegraphics[width=0.8\textwidth]{images/vista_paciente.png}
\end{center}

\vspace{0.5cm}

Como última vista tenemos la de las consultas, está es algo diferente ya que tiene los dos campos que toman un ID, el de médico y el de paciente, junto con una fecha.

\vspace{0.5cm}

\begin{center}
    \includegraphics[width=0.8\textwidth]{images/vista_consulta.png}
\end{center}

\vspace{0.5cm}

Con todo esto hecho, podemos probar a crear unos ejemplos y aprovechar para ver los formularios.

Empezaremos creando un médico, podremos indicar los campos ya mencionados anteriormente.

\vspace{0.5cm}

\begin{center}
    \includegraphics[width=0.8\textwidth]{images/creacion_medico1.png}
\end{center}

\vspace{0.5cm}

También en el propio formulario tenemos para añadir consultas directamente a este médico.

\vspace{0.5cm}

\begin{center}
    \includegraphics[width=0.8\textwidth]{images/creacion_medico2.png}
\end{center}

\vspace{0.5cm}

Ahora crearemos un nuevo paciente (con los campos ya mencionados anteriormente):

\vspace{0.5cm}

\begin{center}
    \includegraphics[width=0.8\textwidth]{images/creacion_paciente1.png}
\end{center}

\vspace{0.5cm}

Y al igual que con los médicos podemos indicar una consulta directamente aquí con alguno de los médicos:

\vspace{0.5cm}

\begin{center}
    \includegraphics[width=0.8\textwidth]{images/creacion_paciente2.png}
\end{center}

\vspace{0.5cm}

Por último crearemos una consulta con el médico y paciente que acabamos de crear y le indicaremos un diagnóstico:

\vspace{0.5cm}

\begin{center}
    \includegraphics[width=0.8\textwidth]{images/creacion_consulta.png}
\end{center}

\vspace{0.5cm}


\newpage
\section{Creación de módulo de ciclo formativo}

\textit{Cuando iba a empezar este módulo tuve que eliminar la BBDD que tenía previamente en Odoo porque estaba corrupta y no podía iniciarla, asi que tuve que volver a montar el Odoo (por si hay algunos nombres diferentes)}\\

La última actividad consistirá en la creación de un módulo que represente los estudios de ciclos formativos en un instituto. Usaremos los siguiente modelos:

\begin{itemize}
    \item Ciclo: Representa un ciclo formativo en el instituto.
    \item Módulo: Se relaciona con un ciclo formativo (pertenece), los alumnos (matriculados) y los profesores (que lo imparten).
    \item Alumno: Representa un alumno que se matricula en módulos.
    \item Profesor: Imparte diferentes módulos.
\end{itemize}

Empezaremos montando el módulo base con 'scaffold' como ya explicamos en la actividad anterior y añadiremos los modelos que usemos al init\_\_.py. Ya con la base montada nos pondremos a trabajar.

En este caso vamos a necesitar cuatro modelos (los listados anteriormente) en total. Vamos a empezar por el de los ciclos. 

Para este modelo solo necesitaremos dos cosas, el nombre del ciclo y una relación con los módulos (que será 'One2many' debido a que un ciclo puede tener X módulos)

\vspace{0.5cm}

\begin{lstlisting}[caption=Modelo ciclo]
from odoo import models, fields

class Ciclo(models.Model):
    _name = 'instituto.ciclo'
    _description = 'Ciclos Formativos'

    name = fields.Char(string='Nombre del Ciclo',required=True)
    modulo_ids = fields.One2many('instituto.modulo','ciclo_id',string='Modulos')
\end{lstlisting}

\vspace{0.5cm}

Seguimos con los módulos, para este necesitaremos también un nombre, pero más importante todavía, vamos a referenciar a todos los modelos (tanto el ciclo que ya hemos creado como los dos que nos faltan), 
gestionamos las relaciones de manera que el módulo tenga X alumnos, pero solo pertenece a un ciclo y solo lo imparte un profesor.

\vspace{0.5cm}

\begin{lstlisting}[caption=Modelo módulo]
from odoo import models, fields

class Modulo(models.Model):
    _name = 'instituto.modulo'
    _description = 'Modulo'

    name = fields.Char(string='Nombre del Modulo', required=True)
    ciclo_id = fields.Many2one('instituto.ciclo', string='Ciclo Formativo', required=True)
    alumnos_ids = fields.Many2many('instituto.alumno', string='Alumnos Matriculados')
    profesor_id = fields.Many2one('instituto.profesor', string='Profesor')
\end{lstlisting}

\vspace{0.5cm}

El tercer modelo será el de alumno, que similar al de los ciclos, tendrá un nombre y en este caso una referencia 'Many2many' ya que un alumno tiene X módulos y un módulo tiene X alumnos.

\vspace{0.5cm}

\begin{lstlisting}[caption=Modelo alumno]
from odoo import models, fields

class Alumno(models.Model):
    _name = 'instituto.alumno'
    _description = 'Alumno'

    name = fields.Char(string='Nombre del Alumno',required=True)
    modulo_ids = fields.Many2many('instituto.modulo', string='Modulos del Alumno')
\end{lstlisting}

\vspace{0.5cm}

El último modelo será el de profesor, que es prácticamente igual al de los alumnos, teniendo un nombre y una relación 'Many2many'.

\vspace{0.5cm}

\begin{lstlisting}[caption=Modelo profesor]
from odoo import models, fields

class Profesor(models.Model):
    _name = 'instituto.profesor'
    _description = 'Profesor'

    name = fields.Char(string='Nombre del Profesor',required=True)
    modulo_ids = fields.Many2many('instituto.modulo', string='Modulos del Profesor')
\end{lstlisting}

\vspace{0.5cm}

Ahora que tenemos los modelos creados, haremos las vistas para cada uno de ellos. Las vistas serán básicas, contando con una lista para ver los existentes y un formulario para la creación.

La vista de los ciclos es tal que así, mostrando el nombre:

\vspace{0.5cm}

\begin{center}
    \includegraphics[width=0.8\textwidth]{images/vista_ciclos.png}
\end{center}

\vspace{0.5cm}

La vista de módulos es la más completa, puedes ver el nombre del módulo, el ciclo al que pertenece y el profesor que lo imparte:

\vspace{0.5cm}

\begin{center}
    \includegraphics[width=0.8\textwidth]{images/vista_modulos.png}
\end{center}

\vspace{0.5cm}

En la vista de alumnos podemos ver sus nombres:

\vspace{0.5cm}

\begin{center}
    \includegraphics[width=0.8\textwidth]{images/vista_alumnos.png}
\end{center}

\vspace{0.5cm}

Y para los profesores también podemos ver sus nombres:

\vspace{0.5cm}

\begin{center}
    \includegraphics[width=0.8\textwidth]{images/vista_profesor.png}
\end{center}

\vspace{0.5cm}

Ahora que tenemos las vistas vamos a probar la creación de cada uno de los modelos:

Vamos a empezar creando un ciclo, podremos ponerle un nombre y meterlo directamente en un módulo:

\vspace{0.5cm}

\begin{center}
    \includegraphics[width=0.8\textwidth]{images/creacion_ciclo1.png}
\end{center}

\vspace{0.5cm}

O podemos crear el módulo desde aquí mismo y añadírselo:

\vspace{0.5cm}

\begin{center}
    \includegraphics[width=0.8\textwidth]{images/creacion_ciclo2.png}
\end{center}

\vspace{0.5cm}

Ahora vamos a crear un módulo (que va a ser igual que la creación que acabamos de hacer desde ciclos).

\vspace{0.5cm}

\begin{center}
    \includegraphics[width=0.8\textwidth]{images/creacion_modulo1.png}
\end{center}

\vspace{0.5cm}

Desde aquí podremos crear todo lo necesario si es que ya no están creado previamente, como ciclos, alumnos y profesores.

\vspace{0.5cm}

\begin{center}
    \includegraphics[width=0.8\textwidth]{images/creacion_modulo2.png}
\end{center}

\vspace{0.5cm}

Ahora en alumnos podremos ponerles su nombre, y añadirlos al módulo/s que sea/n necesarios.

\vspace{0.5cm}

\begin{center}
    \includegraphics[width=0.8\textwidth]{images/creacion_alumno.png}
\end{center}

\vspace{0.5cm}

Y por último crearemos un profesor, también con su nombre y los módulos que imparte:

\vspace{0.5cm}

\begin{center}
    \includegraphics[width=0.8\textwidth]{images/creacion_profesor.png}
\end{center}

\vspace{0.5cm}

Ahora que tenemos todos los modelos y vistas nos queda ver la configuración de seguridad. Crearemos un '.xml' en la carpeta \textbf{'security'}.

Dentro de el empezaremos definiendo los dos grupos de usuarios que queremos (profesor y director):

\vspace{0.5cm}

\begin{lstlisting}[caption=Grupos de usuarios]
    <record id="grupo_director" model="res.groups">
        <field name="name">Director</field>
    </record>

    <record id="grupo_profesor" model="res.groups">
        <field name="name">Profesor</field>
    </record>
\end{lstlisting}

\vspace{0.5cm}

Ahora definiremos los permisos propiamente dichos, línea por línea iremos indicando una regla y el modelo al que se le va a aplicar, luego definimos el 'filtro' que queremos 
(basicamente permitir a los profesores ver a sus iguales) y limitamos esa regla al grupo profesor. Para finalizar indicamos los permisos que tendrán (en este caso, solo lectura).

\vspace{0.5cm}

\begin{lstlisting}[caption=Regla de seguridad para profesor]
    <record id="profesor_regla" model="ir.rule">
        <field name="name">Profesor solo lectura de profesores</field>
        <field name="model_id" ref="model_instituto_profesor"/>
        <field name="domain_force">[(1,'=',1)]</field>
        <field name="groups" eval="[(4, ref('ciclos.grupo_profesor'))]"/>
        <field name="perm_read" eval="True"/>
        <field name="perm_write" eval="False"/>
        <field name="perm_create" eval="False"/>
        <field name="perm_unlink" eval="False"/>
    </record>
\end{lstlisting}

\vspace{0.5cm}

Para los directores evitamos añadir otra regla ya que la el predefinido para Odoo será lo que indiquemos en nuestro '.csv' que está en la carpeta 'security:

\vspace{0.5cm}

\begin{lstlisting}[caption=CSV con permisos]
id,name,model_id:id,group_id:id,perm_read,perm_write,perm_create,perm_unlink
access_alumno,access_alumno,model_instituto_alumno,,1,1,1,1
access_modulo,access_modulo,model_instituto_modulo,,1,1,1,1
access_profesor_read,access_profesor_read,model_instituto_profesor,grupo_profesor,1,0,0,0
access_profesor,access_profesor,model_instituto_profesor,grupo_director,1,1,1,1
access_ciclo,access_ciclo,model_instituto_ciclo,,1,1,1,1
\end{lstlisting}

\vspace{0.5cm}

Esto quiere decir que  todos los usuarios (en nuestro caso solo tenemos profesor, director y administrador), tendrán todos los permisos, 
pero con lo anterior en el '.xml' limitamos solo a los profesores

Podemos verlo de manera más visual si creamos un usuario profesor marcando la casilla de 'profesor':

\vspace{0.5cm}

\begin{center}
    \includegraphics[width=0.8\textwidth]{images/creacion_usuario.png}
\end{center}

\vspace{0.5cm}

Podemos acceder a sus reglas de registro y aquí encontraremos nuestra regla 'Profesor solo lectura de profesores' para limitarlo a solo lectura:

\vspace{0.5cm}

\begin{center}
    \includegraphics[width=0.8\textwidth]{images/reglas_profesor.png}
\end{center}

\vspace{0.5cm}

Y si vamos a profesores con este usuario vemos que solo podemos leer los profesor actuales, pero no modificar, crear, o borrar nada.

\vspace{0.5cm}

\begin{center}
    \includegraphics[width=0.8\textwidth]{images/vista_desde_profesor.png}
\end{center}

\vspace{0.5cm}

Ahora probaremos creando un director marcando la casilla correspondiente:

\vspace{0.5cm}

\begin{center}
    \includegraphics[width=0.8\textwidth]{images/creacion_usuario_director.png}
\end{center}

\vspace{0.5cm}

En este caso la regla no aplica para los directores así que no aparecera, pero podemos comprobar que tienen todos los permisos viendo la pestaña de profesores:

\vspace{0.5cm}

\begin{center}
    \includegraphics[width=0.8\textwidth]{images/vista_desde_director.png}
\end{center}

\vspace{0.5cm}

Con la actividad 4 terminada, queda finalizada la práctica de creación de módulos y vistas en Odoo

\end{document}