\documentclass[12pt, a4paper]{article}

\usepackage[utf8]{inputenc}
\usepackage[spanish, es-tabla]{babel}

\usepackage[a4paper, margin=2.5cm]{geometry}
\linespread{1.2}

\usepackage{xcolor}
\usepackage{graphicx}
\usepackage{hyperref}
\usepackage{listings}
\usepackage{booktabs}

\hypersetup{
    colorlinks=true,
    linkcolor=blue!70!black,
    urlcolor=cyan!70!black,
}

\renewcommand{\lstlistingname}{Código}

\lstdefinestyle{mystyle}{
    backgroundcolor=\color{black!5},
    basicstyle=\ttfamily\footnotesize,
    breaklines=true,
    frame=single,
    framerule=0.5pt,
    rulecolor=\color{black!20},
    numbers=left,
    numberstyle=\tiny\color{gray},
    numbersep=5pt,
    captionpos=b,
}
\lstset{style=mystyle}

\begin{document}

\begin{titlepage}
    \centering
    \includegraphics[width=0.5\textwidth]{images/odoo_logo.png}
    
    \vspace{3cm}
    
    \Huge\bfseries
    Documentación Técnica: \\
    Módulos Odoo y Web Controllers
    
    \vfill
    
    \Large
    \textbf{Autor:} Pedro José Meixús Belsol \\
    \vspace{0.5cm}
    \today
\end{titlepage}

\tableofcontents
\newpage

\section{Modificar módulo liga fútbol}
Tomando como base el módulo \texttt{EJ07\-LigaFutbol} que se nos ha proporcionado, vamos a realizar diferentes modificaciones para ampliar su funcionalidad.

\subsection{Reglas de puntuación especiales}
Empezaremos por modificar las reglas de puntuación. Haremos que los partidos con 4 o más goles de diferencia otorguen 4 puntos al equipo ganador y -1 al perdedor.
Para eso crearemos una función en el modelo de 'liga\_partido.py' para que lo calcule, simplemente comparando la diferencia de goles de goles.

\vspace{0.5cm}

\begin{center}
    \includegraphics[width=0.8\textwidth]{images/codigo_puntuacion.png}
\end{center}

\vspace{0.5cm}

Podemos comprobarlo creando un partido con 4 goles de diferencia y viendo la tabla de clasificación.
Ahora mismo esta es la tabla de clasificación:

\vspace{0.5cm}

\begin{center}
    \includegraphics[width=1\textwidth]{images/tabla_puntuacion.png}
\end{center}

\vspace{0.5cm}

Vamos a añadir un partido en el que el equipo Real Madrid gana 6-0.

\vspace{0.5cm}

\begin{center}
    \includegraphics[width=1\textwidth]{images/creacion_partido_puntos.png}
\end{center}

\vspace{0.5cm}

Y ahora la puntuación debería haber aumentando en 4 puntos para el Real Madrid y disminuido en 1 para el Barcelona.

\vspace{0.5cm}

\begin{center}
    \includegraphics[width=1\textwidth]{images/tabla_puntuacion2.png}
\end{center}

\vspace{0.5cm}

\newpage
\subsection{Botones para alterar los goles de los partidos}
Vamos a añadir dos botones en la vista de formulario de los partidos para aumentar o disminuir los goles de todos los equipos locales o visitantes.
Tendremos que añadir dos funciones en el modelo de 'liga\_partido.py' para que lo hagan.

\vspace{0.5cm}

\begin{center}
    \includegraphics[width=1\textwidth]{images/codigo_sumar_goles.png}
\end{center}

\vspace{0.5cm}

Vamos a probar si funciona, primero veremos los resultados de los partidos.

\vspace{0.5cm}

\begin{center}
    \includegraphics[width=1\textwidth]{images/resultados_puntos.png}
\end{center}

\vspace{0.5cm}

Ahora accedo a uno de ellos y pulso una vez cada botón.

\vspace{0.5cm}

\begin{center}
    \includegraphics[width=1\textwidth]{images/botones_sumar.png}
\end{center}

\vspace{0.5cm}

Y ahora volvemos a ver los resultados de los partidos para comprobar que se han modificado correctamente.

\vspace{0.5cm}

\begin{center}
    \includegraphics[width=1\textwidth]{images/resultados_puntos2.png}
\end{center}

\vspace{0.5cm}

Para probar que la clasificación se recalcula, voy a sumarle dos veces a los visitantes y al ver la clasificación podemos ver que ha cambiado.

\vspace{0.5cm}

\begin{center}
    \includegraphics[width=1\textwidth]{images/tabla_puntuacion3.png}
\end{center}

\vspace{0.5cm}

\newpage
\subsection{Web Controller para eliminar empates}

Vamos a añadir un Web Controller que nos permita eliminar todos los partidos que hayan terminado en empate.
Primero añadiremos un '@http.route' con una función que buscara los partidos terminados en empate y los eliminará.

\vspace{0.5cm}

\begin{center}
    \includegraphics[width=0.8\textwidth]{images/codigo_empate.png}
\end{center}

\vspace{0.5cm}

Ahora crearemos un partido que termine en empate, si no tenemos alguno lo crearemos como en mi caso.

\vspace{0.5cm}

\begin{center}
    \includegraphics[width=1\textwidth]{images/con_empate.png}
\end{center}

\vspace{0.5cm}

Teniendo uno por lo menos iremos a la URL del Web Controller que hemos creado ('http://localhost:8069/eliminarempates') 
y eliminará automaticamente los partidos en empate.

\vspace{0.5cm}

\begin{center}
    \includegraphics[width=1\textwidth]{images/borrar_empate.png}
\end{center}

\vspace{0.5cm}

Y si volvemos a la vista de partidos podemos ver que el partido en empate ha sido eliminado.

\vspace{0.5cm}

\begin{center}
    \includegraphics[width=1\textwidth]{images/sin_empate.png}
\end{center}

\vspace{0.5cm}

\newpage
\subsection{Informe PDF por partido}
No he conseguido hacer funcionar esta parte.

*Revisar en la sección de problemas encontrados al final del documento.

\newpage
\subsection{Wizard para crear nuevos partidos}
Vamos a crear un Wizard que nos permita crear nuevos partidos de una forma más sencilla.
Para eso crearemos dos archivos nuevos, uno para el modelo del Wizard y otro para la vista dentro de la carpeta 'wizard'.

En el modelo crearemos los campos necesarios (y sus relaciones) para crear un partido y una función que lo cree.

\vspace{0.5cm}

\begin{center}
    \includegraphics[width=0.8\textwidth]{images/codigo_wizard.png}
\end{center}

\vspace{0.5cm}

Y en la vista crearemos el formulario para el Wizard.

\vspace{0.5cm}

\begin{center}
    \includegraphics[width=0.8\textwidth]{images/codigo_wizard_xml.png}
\end{center}

\vspace{0.5cm}

Con esto ya hecho, añadiremos la ruta del xml al '\_\_manifest\_\_.py' para que Odoo los reconozca. 
Y ahora veremos un nuevo botón en el menú superior para abrir el Wizard.

\vspace{0.5cm}

\begin{center}
    \includegraphics[width=1\textwidth]{images/boton_wizard.png}
\end{center}

\vspace{0.5cm}

Al pulsarlo se abrirá el Wizard que hemos creado y podremos añadir un nuevo partido.

\vspace{0.5cm}

\begin{center}
    \includegraphics[width=1\textwidth]{images/wizard_funcionando.png}
\end{center}

\vspace{0.5cm}

\newpage
\subsection{Vista Graph para goles de equipos locales}
Vamos a añadir una vista Graph que nos muestre los goles totales de los equipos locales.

Para esto he añadido la vista Graph en el archivo XML de 'liga\_partido.xml'.

\vspace{0.5cm}

\begin{center}
    \includegraphics[width=0.8\textwidth]{images/codigo_graph.png}
\end{center}

\vspace{0.5cm}

Y también he añadido la acción para que se una de las vistas posibles.

\vspace{0.5cm}

\begin{center}
    \includegraphics[width=1\textwidth]{images/codigo_graph2.png}
\end{center}

\vspace{0.5cm}

Ahora si vamos a la vista de partidos podemos seleccionar la vista Graph y ver los goles totales de los equipos locales.

\vspace{0.5cm}

\begin{center}
    \includegraphics[width=1\textwidth]{images/vista_graph.png}
\end{center}

\vspace{0.5cm}

\newpage
\section{Bot de Telegram conectado a la API REST}

En esta actividad vamos a crear un bot de Telegram que se conecte a la API REST que creamos 
en la actividad anterior. Usaremos la librería 'python-telegram-bot' para crear el bot y la 
librería 'requests' para hacer las peticiones HTTP a la API REST. Dependiendo de la orden que se 
envíe se realizarán diferentes peticiones (GET, POST, PUT o DELETE) a la API, Odoo procesara esa 
solicitud y el bot devolverá el resultado al usuario.

\subsection{Creación del bot}

Primero de nada, debemos crear el propio bot en telegram. Para eso, abrimos Telegram y buscamos el bot 'BotFather'.
Le enviamos el comando '/newbot' y seguimos las instrucciones para crear nuestro bot.

\vspace{0.5cm}

\begin{center}
    \includegraphics[width=0.8\textwidth]{images/creacion_bot.png}
\end{center}

\vspace{0.5cm}

Nos dará un token que guardaremos y usaremos para conectar nuestro bot con la API de Telegram.

\newpage
\subsection{Creación y configuración de entorno para el bot}

Ahora crearemos un entorno virtual para nuestro bot con el siguiente comando.

\vspace{0.5cm}

\begin{center}
    \includegraphics[width=1\textwidth]{images/creacion_entorno.png}
\end{center}

\vspace{0.5cm}

Accedemos al entorno virtual e instalaremos las librerías necesarias con pip(el python-telegram-bot y requests).

\vspace{0.5cm}

\begin{center}
    \includegraphics[width=1\textwidth]{images/instalacion_bot.png}
\end{center}

\vspace{0.5cm}

\newpage
\subsection{Desarrollo}

Posteriormente crearemos un archivo 'bot\_socios.py' y añadiremos el código necesario para crear el bot. 
Aquí es donde deberemos usar por un lado el token que nos dio BotFather y por otro la URL de nuestra API REST.

\vspace{0.5cm}

\begin{center}
    \includegraphics[width=0.8\textwidth]{images/codigo_token.png}
\end{center}

\vspace{0.5cm}

También deberemos indicar las funciones para interactuar con la API REST (crear, modificar, consultar y borrar socios).

\vspace{0.5cm}

\begin{center}
    \includegraphics[width=0.8\textwidth]{images/codigo_bot.png}
\end{center}

\vspace{0.5cm}

Las funciones anteriores se usarán en la función principal del bot (handler), que se encargará de manejar los mensajes recibidos y llamar a las funciones correspondientes según el comando recibido.

\vspace{0.5cm}

\begin{center}
    \includegraphics[width=0.8\textwidth]{images/codigo_botFinal.png}
\end{center}

\vspace{0.5cm}

\newpage
\subsection{Arranque y funcionamiento}

Con todo esto hecho podemos iniciar nuestro bot con el siguiente comando.

\vspace{0.5cm}

\begin{center}
    \includegraphics[width=1\textwidth]{images/iniciarBot.png}
\end{center}

\vspace{0.5cm}

Desde dentro usaremos el comando 'python' junto con el nombre del archivo para iniciarlo y si todo esta correcto iniciará sin problema.

\vspace{0.5cm}

\begin{center}
    \includegraphics[width=1\textwidth]{images/botIniciado.png}
\end{center}

\vspace{0.5cm}

Ahora nos quedará probar los comandos uno por uno desde Telegram y ver si los cambios se reflejan en Odoo. Empecemos creando un socio.

\vspace{0.5cm}

\begin{center}
    \includegraphics[width=0.8\textwidth]{images/comandoBot1.png}
\end{center}

\vspace{0.5cm}

Y comprobamos en Odoo que se ha creado correctamente.

\vspace{0.5cm}

\begin{center}
    \includegraphics[width=1\textwidth]{images/botPrueba1.png}
\end{center}

\vspace{0.5cm}

Probemos a modificarlo.

\vspace{0.5cm}

\begin{center}
    \includegraphics[width=0.8\textwidth]{images/comandoBot2.png}
\end{center}

\vspace{0.5cm}

Revisamos en Odoo que el apellido ha sido modificado.

\vspace{0.5cm}

\begin{center}
    \includegraphics[width=1\textwidth]{images/botPrueba2.png}
\end{center}

\vspace{0.5cm}

Ahora borraremos el socio que hemos creado.

\vspace{0.5cm}

\begin{center}
    \includegraphics[width=0.8\textwidth]{images/comandoBot3.png}
\end{center}

\vspace{0.5cm}

Y comprobamos en Odoo que ha sido eliminado.

\vspace{0.5cm}

\begin{center}
    \includegraphics[width=1\textwidth]{images/botPrueba3.png}
\end{center}

\vspace{0.5cm}

Para terminar las operaciones probaremos a consultar un socio (en este caso el que tenga el numero de socio 3).

\vspace{0.5cm}

\begin{center}
    \includegraphics[width=0.8\textwidth]{images/comandoBot4.png}
\end{center}

\vspace{0.5cm}

Por último, comprobamos que al mandar cualquier otro mensaje el bot responde 'Orden no soportada'.

\vspace{0.5cm}

\begin{center}
    \includegraphics[width=0.8\textwidth]{images/comandoBot5.png}
\end{center}

\vspace{0.5cm}

Si todo ha funcionado correctamente, nuestro bot estará terminado.

\newpage
\section{Generación de imágenes aleatorias con Web Controllers}
Para esta parte de la práctica, vamos a crear un Web Controller que genere imágenes aleatorias tomando como base el módulo 
'EJ09-GenerarBarcode' proporcionado.

\subsection{Instalaciones necesarias}
Primero deberemos acceder a nuestro contenedor de Odoo, para buscarlo usaremos el comando 'docker ps'.

\vspace{0.5cm}

\begin{center}
    \includegraphics[width=1\textwidth]{images/contenedores.png}
\end{center}

\vspace{0.5cm}

Esto nos mostrará los contenedores activos, buscaremos el ID del  contenedor de Odoo y accederemos a el con el siguiente comando.

\vspace{0.5cm}

\begin{center}
    \includegraphics[width=1\textwidth]{images/bash.png}
\end{center}

\vspace{0.5cm}

Una vez dentro usaremos el comando 'pip list' para ver las librerías instaladas.

\vspace{0.5cm}

\begin{center}
    \includegraphics[width=1\textwidth]{images/librerias.png}
\end{center}

\vspace{0.5cm}

En cualquier caso, haremos un 'apt-get update' para actualizar lo necesario.

\vspace{0.5cm}

\begin{center}
    \includegraphics[width=1\textwidth]{images/actualizar.png}
\end{center}

\vspace{0.5cm}

Ahora instalaremos las librerías que vamos a necesitar con los siguiente comandos.

\vspace{0.5cm}

\begin{center}
    \includegraphics[width=1\textwidth]{images/instalacion_python.png}
\end{center}

\vspace{0.5cm}

Hecho esto, reiniciamos el contenedor para que los cambios tengan efecto. Y activamos el módulo en Odoo.

\vspace{0.5cm}

\begin{center}
    \includegraphics[width=0.8\textwidth]{images/activar_modulo.png}
\end{center}

\vspace{0.5cm}

\newpage
\subsection{Prueba Web Controller de código de barras}

Ahora podemos probar el Web Controller del código de barras que viene por defecto accediendo a la URL 'http://localhost:8069/generador/123456789012'.

\vspace{0.5cm}

\begin{center}
    \includegraphics[width=1\textwidth]{images/barras.png}
\end{center}

\vspace{0.5cm}

\newpage
\subsection{Creación de Web Controller para imágenes aleatorias}

Vamos ahora entonces a añadir nuestro propio Web Controller para generar imágenes aleatorias. Añadiremos un archivo en mi caso 
'generarimagen.py' y añadiremos el siguiente código, que generará pixeles de colores aleatorios con el alto y ancho que indiquemos.

La imagen se genera en formato PNG y se devolverá como respuesta al acceder a la URL.

\vspace{0.5cm}

\begin{center}
    \includegraphics[width=0.8\textwidth]{images/codigo_imagen.png}
\end{center}

\vspace{0.5cm}

Con esto hecho, actualizamos el módulo y probamos el Web Controller accediendo a la URL 'http://localhost:8069/imagen/aleatoria?ancho=300\&alto=300' 
(indicando el alto y ancho deseados).

\vspace{0.5cm}

\begin{center}
    \includegraphics[width=1\textwidth]{images/imagen.png}
\end{center}

\vspace{0.5cm}

\newpage
\section{Problemas encontrados}

Durante la realización de esta práctica he tenido algunos problemas básicos como sintaxis de Python o permisos mal configurados.

*Pero el principal problema lo he tenido en el apartado 4 de la primera actividad, el informe PDF para los partidos.
Al intentar generar el PDF, me da un error de 'External ID not found'.

\vspace{0.5cm}

\begin{center}
    \includegraphics[width=1\textwidth]{images/error_pdf.png}
\end{center}

\vspace{0.5cm}

Por la experiencia que tengo con Odoo y lo que he investigado, debería ser un problema con la referencia 
(que la hubiera escrito mal) o que no haya declarado el XML en el manifest y Odoo no pueda verlo, pero aún 
revisandolo varias veces no he conseguido encontrar el error, por lo que no he podido completar esa parte de la práctica.

Algunos problemas menores que he tenido también fueron con la creación del bot de Telegram, la recalculación de las puntuaciones de 
los partidos y los botones para alterar los goles, pero no espcialmente destacables

\newpage
\section{Conclusión}
Esta práctica me ha servido para desenvolverme más en la creación y modificación de módulos en Odoo, 
la creación de Web Controllers, la interacción de una API REST con otro servicio diferente (en este caso, el bot de Telegram).

\end{document}